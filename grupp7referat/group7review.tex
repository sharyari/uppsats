\documentclass[a4paper,10pt]{article}
\usepackage{fullpage}
\usepackage[british]{babel}
\usepackage[T1]{fontenc}
\usepackage{amsmath}
\usepackage{amssymb}
\usepackage[T1]{fontenc}
\usepackage[utf8]{inputenc}
%\usepackage{amsthm} \newtheorem{theorem}{Theorem}
\usepackage{color}

\usepackage{float}
\usepackage{enumerate}
\usepackage{caption}
\DeclareCaptionFont{white}{\color{white}}
\DeclareCaptionFormat{listing}{\colorbox{gray}{\parbox{\textwidth}{#1#2#3}}}
\captionsetup[lstlisting]{format=listing,labelfont=white,textfont=white}

\usepackage{alltt}
\usepackage{listings}
 \usepackage{aeguill}
\usepackage{dsfont}
%\usepackage{algorithm}
\usepackage[noend]{algorithm2e}
%\usepackage{algorithmicx}
\usepackage{subfig}
\lstset{% parameters for all code listings
language=Python,
frame=single,
basicstyle=\small, % nothing smaller than \footnotesize, please
tabsize=2,
numbers=left,
% framexleftmargin=2em, % extend frame to include line numbers
%xrightmargin=2em, % extra space to fit 79 characters
breaklines=true,
breakatwhitespace=true,
prebreak={/},
captionpos=b,
columns=fullflexible,
escapeinside={\#*}{\^^M}
}


% Alter some LaTeX defaults for better treatment of figures:
    % See p.105 of "TeX Unbound" for suggested values.
    % See pp. 199-200 of Lamport's "LaTeX" book for details.
    % General parameters, for ALL pages:
    \renewcommand{\topfraction}{0.9}	% max fraction of floats at top
    \renewcommand{\bottomfraction}{0.8}	% max fraction of floats at bottom
    % Parameters for TEXT pages (not float pages):
    \setcounter{topnumber}{2}
    \setcounter{bottomnumber}{2}
    \setcounter{totalnumber}{4} % 2 may work better
    \setcounter{dbltopnumber}{2} % for 2-column pages
    \renewcommand{\dbltopfraction}{0.9}	% fit big float above 2-col. text
    \renewcommand{\textfraction}{0.07}	% allow minimal text w. figs
    % Parameters for FLOAT pages (not text pages):
    \renewcommand{\floatpagefraction}{0.7}	% require fuller float pages
% N.B.: floatpagefraction MUST be less than topfraction !!
    \renewcommand{\dblfloatpagefraction}{0.7}	% require fuller float pages

% remember to use [htp] or [htpb] for placement


\usepackage{fancyvrb}
%\DefineVerbatimEnvironment{code}{Verbatim}{fontsize=\small}
%\DefineVerbatimEnvironment{example}{Verbatim}{fontsize=\small}

\usepackage{url}
\urldef{\mailsa}\path|josh0151@student.uu.se |
\urldef{\mailsb}\path|bjfo5755@student.uu.se |
\newcommand{\keywords}[1]{\par\addvspace\baselineskip
\noindent\keywordname\enspace\ignorespaces#1}


\usepackage{tikz} \usetikzlibrary{trees}
\usepackage{hyperref} % should always be the last package

% useful colours (use sparingly!):
\newcommand{\blue}[1]{{\color{blue}#1}}
\newcommand{\green}[1]{{\color{green}#1}}
\newcommand{\red}[1]{{\color{red}#1}}

% useful wrappers for algorithmic/Python notation:
\newcommand{\length}[1]{\text{len}(#1)}
\newcommand{\twodots}{\mathinn\usepackage{enumerate}er{\ldotp\ldotp}} % taken from clrscode3e.sty
\newcommand{\Oh}[1]{\mathcal{O}\left(#1\right)}

% useful (wrappers for) math symbols:
\newcommand{\Cardinality}[1]{\left\lvert#1\right\rvert}
%\newcommand{\Cardinality}[1]{\##1}
\newcommand{\Ceiling}[1]{\left\lceil#1\right\rceil}
\newcommand{\Floor}[1]{\left\lfloor#1\right\rfloor}
\newcommand{\Iff}{\Leftrightarrow}
\newcommand{\Implies}{\Rightarrow}
\newcommand{\Intersect}{\cap}
\newcommand{\Sequence}[1]{\left[#1\right]}
\newcommand{\Set}[1]{\left\{#1\right\}}
\newcommand{\SetComp}[2]{\Set{#1\SuchThat#2}}
\newcommand{\SuchThat}{\mid}
\newcommand{\Tuple}[1]{\langle#1\rangle}
\newcommand{\Union}{\cup}
\usetikzlibrary{positioning,shapes,shadows,arrows}

\usepackage{url}

\pagestyle{empty}

\title{Review of DynaStat - \\
	\emph{Ett system som sparar och sammanställer} \\
	\emph{siffrorna från dina annonser.}\\
	\vspace{3mm} \normalsize Original paper by: E. Eriksson, E. Kajgård, T. Wallenborg}

\author{Bj{\"o}rn Forsberg, Jonathan Sharyari}
\oddsidemargin 0.5in 
\textwidth 5.5in 
\begin{document}


\maketitle


\section{Abstract}

The role of computers is increasing in society whereas the cost of digital screens is decreasing. This has led to digital signage to become a part of the everyday landscape in society. Some companies would not benefit, or cannot affoard, owning their own digital signs and instead rent display time from a supplier. In many cases, such companies are interested in knowing where and when their ads are shown, and if possible who is watching and to what extent. This paper describes a general-purpose system to manage this type of statistical information, DynaStat, which works independent on the digital sign-system it is implemented in. Additionally, the authors have written an module that utilises the DynaStat system, to be used within an already existing system onboard the Arlanda Express trains. The article is yet only a draft, and the results are as of yet not apparent from the study.

\section{Content}
Digital signage is a growing field in marketing, and companies can often be interested in knowing the circumstances in which their ads are shown; the time, the place, the frequency and so on. This demand requires that the signage-system can manage additional statistical information - this is the gap that the described project aims to fill.

The described system is named \emph{DynaStat}. The details of the system are not apparant from the article as of yet, but is claimed to provide a general system for statistical management that can be used by several digital signage-systems. DynaStat is the used by the authors and implemented as a module to the existing \emph{Cviewer} system on the Arlanda Express trains. A requirement for the implementation is that it can be done with minimal change to the existing system, as updating the existing system is difficult or not possible.

Addtionally, the article thoroughly describes the procedure and intent of test-driven development. This section is a good description of the technique and easy to follow, but is not in any way connected to the rest of the article. 

Titeln är lång, och lite otydlig. Vilka "siffror" är det som åsyftas?


\section{Article Quality}
Basha på:
stavfel
upprepningar - särsilt "problemet"-sektionen borde fördelas på slutsats och introduktion. Problemet borde sägas mer koncist och tydligt.
Referenser ok, förutom D. Lager, E. Björklund där man inte ser vad det är. De borde inte refereras till i parenteser - man ska kunna läsa...
Brasklappar om att vi inte kan säga så mycket, för att de inte sagt så mycket.


Inte basha på:
Bra flow i texten.
Ser ut att vara ett bra upplägg, även om det inte är färdigt. [brasklapp 2 alltså].

The article is well-written, using a precise and concise language. Abbreviations are avoided unless previously defined and the misspellings in the text are presumably due to the fact that the article has been digitalized from paper. The topic describes the content well and an idea of the content can be formed reading only the title and the abstract. The abstract says little about the drawn results, and the usefulness of the developed system. This reflects the greater flaws in the article as a whole; the result section is not informative enough and no analysis of the usefulness of the system is presented. Although a few alternative systems were mentioned in the article, it would have been preferable if comparison testing had been performed in order to assess the quality of the developed system. From the article it is not indicated if the system is ready to be deployed or if further work is needed.

The presentation of previous related work is relatively rigorous in this paper. Design choices are justified by results of previous research, and in most cases alternative methods are also presented together with a motivation behind the choice made. The same scientific quality is unfortunately not provided in the presentation of results, as it is not backed with measurement data. It is therefore difficult to say whether the conclusions are reasonable or just a matter of opinion.


From this it is clear that the main focus of their project has been the actual implementation of the interface, which is reflected by the detailed technical specifications provided. The text body is well structured with informative section titles, allowing the reader to follow the text in a natural way. 


\section{Summary}
[Här är brasklapp 3 om att vi skriver skit för att de skrev mer skit].
The article describes an inexpensive and easy to use system for controlling a computer with eye movements. The system doesn't use any wearable equipment, instead it uses cameras that pinpoint the position of the eyes and nosetip. This information is then used to guide the mouse cursor on the computer, using eye blinks to emulate "clicks". The methodology for deciding if the system is effective is well described and although no data is presented. The input technique described has already been described in earlier work, and with the given information we cannot determine whether the system is better than those earlier described, other than the fact that it is claimed to be effective by the authors themselves.

In general we found this article well-written and helpful as it gave us an good overall understanding of recent research on this subject, and several references to other articles that might be of interest. We believe this article could be of interest to others interested in the topic of facial-feature based interfaces, and we would recommend the interested reader to read at least the introduction section for references and ideas.


\end{document}