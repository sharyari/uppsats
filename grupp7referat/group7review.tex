\documentclass[a4paper,10pt]{article}
\usepackage{fullpage}
\usepackage[british]{babel}
\usepackage[T1]{fontenc}
\usepackage{amsmath}
\usepackage{amssymb}
\usepackage[T1]{fontenc}
\usepackage[utf8]{inputenc}
%\usepackage{amsthm} \newtheorem{theorem}{Theorem}
\usepackage{color}

\usepackage{float}
\usepackage{enumerate}
\usepackage{caption}
\DeclareCaptionFont{white}{\color{white}}
\DeclareCaptionFormat{listing}{\colorbox{gray}{\parbox{\textwidth}{#1#2#3}}}
\captionsetup[lstlisting]{format=listing,labelfont=white,textfont=white}

\usepackage{todonotes}
\usepackage{alltt}
\usepackage{listings}
 \usepackage{aeguill}
\usepackage{dsfont}
%\usepackage{algorithm}
\usepackage[noend]{algorithm2e}
%\usepackage{algorithmicx}
\usepackage{subfig}
\lstset{% parameters for all code listings
language=Python,
frame=single,
basicstyle=\small, % nothing smaller than \footnotesize, please
tabsize=2,
numbers=left,
% framexleftmargin=2em, % extend frame to include line numbers
%xrightmargin=2em, % extra space to fit 79 characters
breaklines=true,
breakatwhitespace=true,
prebreak={/},
captionpos=b,
columns=fullflexible,
escapeinside={\#*}{\^^M}
}


% Alter some LaTeX defaults for better treatment of figures:
    % See p.105 of "TeX Unbound" for suggested values.
    % See pp. 199-200 of Lamport's "LaTeX" book for details.
    % General parameters, for ALL pages:
    \renewcommand{\topfraction}{0.9}	% max fraction of floats at top
    \renewcommand{\bottomfraction}{0.8}	% max fraction of floats at bottom
    % Parameters for TEXT pages (not float pages):
    \setcounter{topnumber}{2}
    \setcounter{bottomnumber}{2}
    \setcounter{totalnumber}{4} % 2 may work better
    \setcounter{dbltopnumber}{2} % for 2-column pages
    \renewcommand{\dbltopfraction}{0.9}	% fit big float above 2-col. text
    \renewcommand{\textfraction}{0.07}	% allow minimal text w. figs
    % Parameters for FLOAT pages (not text pages):
    \renewcommand{\floatpagefraction}{0.7}	% require fuller float pages
% N.B.: floatpagefraction MUST be less than topfraction !!
    \renewcommand{\dblfloatpagefraction}{0.7}	% require fuller float pages

% remember to use [htp] or [htpb] for placement


\usepackage{fancyvrb}
%\DefineVerbatimEnvironment{code}{Verbatim}{fontsize=\small}
%\DefineVerbatimEnvironment{example}{Verbatim}{fontsize=\small}

\usepackage{url}
\urldef{\mailsa}\path|josh0151@student.uu.se |
\urldef{\mailsb}\path|bjfo5755@student.uu.se |
\newcommand{\keywords}[1]{\par\addvspace\baselineskip
\noindent\keywordname\enspace\ignorespaces#1}


\usepackage{tikz} \usetikzlibrary{trees}
\usepackage{hyperref} % should always be the last package

% useful colours (use sparingly!):
\newcommand{\blue}[1]{{\color{blue}#1}}
\newcommand{\green}[1]{{\color{green}#1}}
\newcommand{\red}[1]{{\color{red}#1}}

% useful wrappers for algorithmic/Python notation:
\newcommand{\length}[1]{\text{len}(#1)}
\newcommand{\twodots}{\mathinn\usepackage{enumerate}er{\ldotp\ldotp}} % taken from clrscode3e.sty
\newcommand{\Oh}[1]{\mathcal{O}\left(#1\right)}

% useful (wrappers for) math symbols:
\newcommand{\Cardinality}[1]{\left\lvert#1\right\rvert}
%\newcommand{\Cardinality}[1]{\##1}
\newcommand{\Ceiling}[1]{\left\lceil#1\right\rceil}
\newcommand{\Floor}[1]{\left\lfloor#1\right\rfloor}
\newcommand{\Iff}{\Leftrightarrow}
\newcommand{\Implies}{\Rightarrow}
\newcommand{\Intersect}{\cap}
\newcommand{\Sequence}[1]{\left[#1\right]}
\newcommand{\Set}[1]{\left\{#1\right\}}
\newcommand{\SetComp}[2]{\Set{#1\SuchThat#2}}
\newcommand{\SuchThat}{\mid}
\newcommand{\Tuple}[1]{\langle#1\rangle}
\newcommand{\Union}{\cup}
\usetikzlibrary{positioning,shapes,shadows,arrows}

\usepackage{url}

\pagestyle{empty}

\title{Review of DynaStat - \\
	\emph{Ett system som sparar och sammanställer} \\
	\emph{siffrorna från dina annonser.}\\
	\vspace{3mm} \normalsize Original paper by: E. Eriksson, E. Kajgård, T. Wallenborg}

\author{Bj{\"o}rn Forsberg, Jonathan Sharyari}
\oddsidemargin 0.5in 
\textwidth 5.5in 
\begin{document}


\maketitle


\section{Abstract}

The role of computers is increasing in society, while the cost of digital displays are decreasing. This has led to digital signage to become a part of the everyday landscape in many parts of society. Some companies want to use this technology for advertisment, but would not benefit from owning their own digital signs. These companies may instead rent display time from a supplier. In many cases, the advertisers are interested in knowing where and when their ads are shown, and if possible who is watching and to what extent. This paper describes a general-purpose system, DynaStat, that manages this type of statistical information. It works independent of which digital sign-system it is implemented in. Additionally, the authors have written an module that utilises the DynaStat system, to be used within an already existing system on board the Arlanda Express trains. The article is yet only a draft, and the results are as of yet not apparent from the study.

\section{Content}
Digital signage is a growing field in marketing, and advertisers are often interested in knowing the circumstances in which their ads are shown; the time, place, frequency, et cetera. This demand requires that the signage-system can manage additional statistical information - this is the gap that the described project aims to fill.

The described system is named \emph{DynaStat}. The details of the system are not apparent from the article as of yet, but is claimed to provide a general system for statistical management that can be used by several digital signage-systems. DynaStat is the used by \todo{va?}the authors and implemented as a module to the existing \emph{Cviewer} system on the Arlanda Express trains. A requirement for the implementation is that it can be done with minimal change to the existing system, as updating the existing system is difficult or not possible.

Additionally, the article thoroughly describes the procedure and intent of \emph{test-driven development} (TDD). In TDD, the developers first define a set of test cases and their desired result. The program is considered to be ready when all test cases give the desired result when executed. The developer can then decide to create a new set of test cases, and develop the application further. This style of development has been shown to decrease the development speed, but increase the coverage of the program during testing.


\section{Article Quality}
This article is still a draft, still missing an abstract and the results of the project. There are a few minor misspellings in the text, which is understandable as the text has yet to be completed and proof-read. There are many repetitions within the text, mainly in the section ``problemet''. This section does not actually describe the problem the authors are trying to solve, as the name of the section implies. A suggestion would be to move most of the information in that section to the introduction or to the body of the report, and keep the problem formulation concise and precise.

A section in the article describes test-driven development. The section is well-written and serves as a good explanation of the main principles of the technique, but it is not apparent from the text how it is connected to the project itself. The authors do not present any benefits of using TDD in this particular project. The reader also has to assume that the authors have used this development style, as it is not clearly stated in the paper.

The references in the article are mainly from respected journals, and well-formatted with the exception of one reference (D. Lager, E. Björklund). The referencing style in the text can at times break the overall good text flow of the article when references are written within parenthesis. This happens when the text continues \emph{into} the parenthesis, going against the norm that a sentence can be read even if the parenthesis is skipped.

It is unclear in the title of the article which \emph{numbers} are meant, but the style of the title in general is good, as it engages the reader from the start.


\section{Summary}
This article describes a system for managing statistical information for digital signs, DynaStat. It does not rely on a specific system to be used, but can be implemented to work with various methods of digital sign control, therefore DynaStat can be considered to be a general\todo{general. oklart?} system. The type of information can be more or less important for the companies using digital signage for advertisment, depending on the nature of the product being advertised.

The reviewed paper is still a draft on an early stage, and it is difficult to determine what the results are, and the quality of the results. It is indicated, but not clearly stated, that the system has already been successfully developed. It is also left to the reader to guess whether this system is actually going to be used on the mentioned Arlanda Express trains or not, and if so, what the connection between the authors and the company is.

The article has a good text flow and apart from some minor spelling errors, is very well written. It discusses an underlying problem that motivates this project, and presents a solution. We believe that it is still far to early to publish this article, but once all sections are finished it has every FÖRUTSÄTTNING\todo{ord} to be published.

\todo{FIXAT?: In general we found this article well-written and helpful as it gave us an good overall understanding of recent research on this subject, and several references to other articles that might be of interest. We believe this article could be of interest to others interested in the topic of facial-feature based interfaces, and we would recommend the interested reader to read at least the introduction section for references and ideas.}


\end{document}