\section{Current}
% Kolla upp hur utgågna patent på current devices spelar in, det verkar vara en faktor på åtminstånde touch screen, kan säkert vara intressant på voice också.
% Brasklapp
% Fundamentals: Asmycket referenser i kapitel 1.4


Several input devices in this paper such as the keyboard and the mouse, are still wildely used but can be seen as historical devices. This is justified by the fact that they remain largely unchainged - if not in their physical design then in the fashion they are used [©itation needed]. On the other hand, a number of other input/output devices that can be seen as to be recent have a long history behind them. taken a larger and larger market share. Although \emph{touchpads}, \emph{touchscreens} and \emph{speak recognition} techniques were researched as far back as the 1960's \cite{buxton}\cite{shoebox}, they have only now began to see commercial use and have likely still not reached their "peak". 


\subsection{Touch devices}
Touchscreen devices have become increasingly popular during the last years, through the breakthrough of \emph{smart devices} (i.e. smart phones and tablets) running specialized operating systems such as the Apple's iOS and Google's Android. The technology used in these devices however, have evolved over over a much longer time.

One of the first mentions of touchscreens was by E.A. Johnson in 1965, describing a touchscreen and identifying some potential uses in a short article, less than a single page\cite{4205802}. Some of the earliest developed systems came in the beginning of the 1970's. Early prototypes are the PLATO \cite{buxton} (1972) and a transparant touchscreen was developed and put to use by CERN in 1973\cite{cern}.

%These devices used METOD - PROBLEM MED DETTA - BESKRIVA NÄSTA GENERATION. (DET SKA HANDLA OM LJUS HÄR, JAG REFERERAR TILL DETTA OSKRIVNA STYCKE SENARE.

The initial uses of touchscreens were point-to-select systems used in for example ATM machines or cash registers in restaurants and were relatively non-demanding on the screen performance\cite{buxton}. But there were also early attemps to make \emph{PDAs}\footnote{Personal Digital Assistant}. The first products to enter the market were the Palm Zoomer and the IBM Simon. The Simon was a mobile phone without any physical buttons, having only a touchscreen as its working area. But beyond regular telephone capabilities, it could also manage information such as a telephone book and be used for drawing and taking notes. Due to this, it is sometimes refered to as the first \emph{smartphone}\cite{buxton}.

Smartphones have seen much development since the release of Simon, and today (2013) it is estimated that more than a billion people use smartphones\cite{billion1}\cite{billion2}. [KORT STYCKE MED REFERENS SOM SÄGER ATT DET ÄR PGA ANDROID OCH IOS]. 

This increase in the number of touch-based devices has changed the way in which we interact with computers. STYCKE SOM PRATAR OM VAD SKILLNADEN ÄR: SVÅRARE ATT ANVÄNDA TANGENTBORD, LÄTTARE ATT GÖRA ANNAT SOM ATT TEX LÄSA BÖCKER, KÖPA SAKER MM.

%Multitouch. Vad är det, och varför?


\subsection{Voice recognition}

Voice recognition is another input method that has become increasingly popular during the last 13 years. This input method uses voice to text methods (bättre namn?) by analyzing input from the microphone. The text this method gives can be parsed by the computer to perform a predefined action depending on the program. Microsoft has recently presented a Spoken english to synthesised speech chinese translator. 

\subsection{Discussion}

% en diskussion och analys som sammanfattar båda ovanstående kategorier.