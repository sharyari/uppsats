% !TEX root = rapport.tex

\begin{abstract}
This paper reviews the history, current state, and future of input and output methods, and the factors leading up to their invention. It hopes to explain what the current trends in development are and what new applications they may allow.

The history of a selected set of computer interaction devices are summarised, from punch cards and the mouse and keyboard, and goes on to investigate the devices in popular use today. Using the knowledge about the factors that lead up to the creation of previous and contemporary devices, the paper goes on to present two possible future methods that follow the same trend of development: gesture control and brain-computer interfaces. A number of solutions aimed towards those in need of assistive devices are also covered.

The report concludes that the interaction method development has become highly user-centered, and goes towards a higher level of abstraction between the user and the machine. It also shows that advances in technology not necessarily replace the previous technology, but rather broaden the spectra of applications where computers are used.

\end{abstract}

\begin{abstract}
Denna artikel granskar historiska, nuvarande och framtida användargränssnitt, och de faktorer som driver deras utveckling. Artikeln avser att förklara vilka de nuvarande utvecklingstendenserna är, och vilka nya tillämpningar dessa tillåter.

En utvald uppsättning gränssnitts historia sammanfattas, från hålkort till mus och tangentbord, och undersöker de gränssnitt som är i populärt bruk idag. Genom att använda kunskapen om tidigare och nutida enheters utveckling presenteras två möjliga framtida gränssnitt som följer samma utvecklingstrend: gestigenkänning och gränssnitt mellan datorer och den mänskliga hjärnan. Artikeln behandlar även ett antal lösningar för personer med handikapp, i behov av särskilda hjälpmedel.

Rapporten dras slutsatsen att utvecklingen av användargränssnitt i allt högre grad blivit användarcentrerad, och eftersträvar en högre abstraktionsnivå mellan dator och människa. Den visar också att de nya gränssnitten inte nödvändigtvis ersätter de tidigare, utan att de snarare kompletterar varandra och breddar uppsättningen användargränssnitt som används.

\end{abstract}
