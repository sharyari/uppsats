% !TEX root = rapport.tex

\section{Problem Formulation and Purpose}
It is apparent that the history of the modern day computer is relatively short and that it has seen rapid development since their introduction. During this time, a broad variety of in- and output devices and methods have been developed, and likewise, a broad range of methods are currently under development. In this maze of devices and approaches, it can be difficult to grasp how these changes lead up to the contemporary methods in use today, and what can be expected even in the near future. The purpose of this paper is to stake out the paths leading up to some of the devices used today, and to try to look ahead at where the paths might lead us in the future.

This also raises some questions on what factors lead up to new devices and methods being popularised. Does new ways of utilising computers lead to new interaction methods? Or does novel interaction methods enable us to use computers in new ways, and if so, how can we expect computers to be applied in the future?

Besides the familiar methods and devices for interaction, there are also special-purpose devices designed towards those who for some reason cannot use the traditional methods of interaction. As the role of the computer in modern day society is increasing, in some cases having become a necessity, it is an important issue to ensure that everyone is included in this development. What alternatives are in existence and how comparable they are in comparison to the traditional devices? Does the rapid development lead to assistive devices falling behind the development, or does it in fact mean the opposite?

