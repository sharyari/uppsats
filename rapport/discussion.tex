% !TEX root = rapport.tex

\section{Results}

%history:

Computer input devices have evolved over a long time, and their introductions were results of very different ideas. The earliest devices were not custom-designed to be used by computers nor to make computers user friendly. Instead they were taken directly from previously existing technologies, and modified in order to function within their new setting. The computer input devices in use, and in development, at the time of this study are designed to make human interaction with computers as simple and easy as possible.

The keyboard-based terminal was the first input device specifically designed for computers. It removed many of the earlier problems associated with programming and running calculations on the computers of that time. While this device was very different from the earlier input devices, the underlying technology was still based on the previous input systems, punch cards. It abstracted away a lot of the technicalities that unnecessarily slowed the end user down.

The pointing devices on the other hand were different. Their introduction was not related to the previous systems, but were instead a result of the introduction of bitmap displays capable of displaying data in a different way than previous computer displays. The need for an input device that handled the new information was needed. blablabla? \todo{skriv nåt bra}.

The touch screens are closely related to the screen/pointing device approach to graphical user interfaces, but were introduced to give the user a more direct way of interacting with the system. The early systems were very simple, while later have multi-touch and better precision, having come so far that they can now be used independently from the traditional computer terminal. \todo{vet inte vad jag vill säga}

%Här är ett stycke som wrappar ihop de fyra tidigare. Den säger att fokus blivit mer och mer på användarvänligheten, och att de hör ihop. Man lärde sig att skiva text smidigt, men ville kunna lätt byta mellan olika "fönster" och flytta pekarens plats lättare. Touchpad, för att musen var för klumpig på laptop, touchscreen för att kunna ha bättre kontroll eller whatever. Jag vet inte riktigt varför man ville ha skiten, men vi skriver något.

%current:

The two current input methods discussed in this paper, speech recognition and touch screens, are very different in their implementations, but have similar undelying ideas, the idea is to remove the hurdles in the interaction between man and machine. The touch screen has evolved as the next step from computer-mouse-keyboard interaction to direct interaction with the data the user is interested in. This makes the interaction much more direct and the learning curve for users is drastically reduced since the computer interacts with the user in the same way a physical object would do.

Speech recognition fulfills the same goal in easing the interraction between human and computer, but instead of doing it physically it is done by making the computer react to the spoken language, much in the same way the user would speak to another human. By making the computer understand the user on the user's terms instead of the computer's, one of the previously most distinct barriers of computer use is removed. The development of speech recognition systems has had a focus much more on cognitive psychology and mathematic models than on the computer science field, which reflects this change of focus.[reference?]

%special:

Computer interaction for people with different kinds of handicaps require different methods depending on the nature of the disability. The computer output commonly uses screens and status LED:s, which only poses a problem for users with low eye sight or blindness. In some cases, the use of sound output poses a problem for hearing impaired users, however these can often be solved with the help of visual cues [HITTA REFERENCE]. 

The input devices are much more diverse in their designs, because they need to adress a larger set of handicaps. Two main categories of special-purpose input methods are devices to aid persons with reduced motor skills, and persons with reduced or no eye-sight.

For persons with reduced motor skills there are different approaches depending on how much of the persons motor skills are reduced. For a person that has reduced motor skills in the hands only, the use of the standard keyboard and mouse can be difficult or impossible. There are solutions where a special designed keyboard and mouse designed for use with the feet can overcome this problem.
For persons that have more extensivly reduced motor skills the use of such devices might also be a problem. There are numerous different approaches to solve this. One of the most interesting is the use of eye movements to steer the computer. This approach makes use of cameras to track the users eye movements and act as a pointing device, which is also used to control an on-screen keyboard. 

For persons that have reduced eye sight the use of braille input can be used to overcome computer use problems. There are numerous approaches with redesigned standard equipment like braille keyboards, and speech feedback. There are also some completely different approaches that are more user centric, as an example the Perkinput system that uses the fingers on a touch surface (touch screen) to input braille characters.

Prata om att inte alla lösningar är hårdvarubaserade, referera till case studyn.

\section{Discussion}

% både tangentbordet och musen kom tidsbesparingskäl, istället för hålla på med hålremsor eller försöka styra grafisk information med tangenter.