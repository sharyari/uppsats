% !TEX root = rapport.tex
\section{Results}

The earliest computer interaction devices were not initially designed for computers nor to make computers user-friendly. Instead existing technologies were directly taken and modified in order to function within their new setting. In contrast, the computer input devices in use at the time of this study are designed to make human interaction with computers as simple and easy as possible.

The typewriter-based \emph{terminal} was one of the first input device specifically designed for computers. It removed many of the earlier problems associated with programming and running calculations on the computers of that time. Although this device was very different from the earlier input devices, the underlying technology was still based on the previous technology, punched cards. It abstracted away technicalities that slowed the user down. In contrast, the development of pointing devices was not related to the previous systems, but instead a result of the introduction of bitmap displays, capable of displaying data in a different way than previous computer displays.

The big difference between these important inventions is the type of impact they had. The terminal allowed the user to communicate with the computer more conveniently than before, but still in the same style. The mouse instead changed computer usage altogether, being instrumental in the window-based systems introduced in the 80's. The impact of touchscreens is in a way similar to that of the mouse, in that it has lead to a new type of graphical user interfaces. Importantly, as these systems are mostly mobile devices, it has not only changed the way we use the computer but also \emph{how} and \emph{when} we use them.

There has been an apparent change in the focus of computer input devices from machine-centric methods, in which the user has to adapt to the computer, to a clear user-centric approach. While the pointing devices were developed to handle input to the new graphical output methods and not specifically to ease the overall computer experience. This laid the ground for the creation of computer interfaces that were more accessible to users without technical background and allowed the computers to move into the homes and offices of regular citizens. This mass-market opportunity created the necessary foundation for focus on more human-centric interfaces that we are still feeling the effects of today.

%current:

The two contemporary input methods discussed in this paper, speech recognition and touchscreens, behave very differently in practice, but they share the same underlying idea - to remove the hurdles in the interaction between man and machine. The touchscreen has evolved as the next step from computer-mouse-keyboard interaction to direct interaction with the data the user is interested in. This allows the user to manipulate the data using fingers and the learning curve for users is drastically reduced when the computer interacts with the user in a more intuitive way.

Speech recognition fulfills the same goal in easing the interraction between human and computer, but instead of doing it physically it is done by reacting to the spoken language in the same way the user would speak to another human. The idea is to make the computer understand the user on the user's own terms. The development of speech recognition systems has required research in cognitive psychology and mathematical (statistical) models rather than in computer science, which reflects the previously mentioned change of focus towards user-friendliness. \todo{Skriv om cognitive psychology i current}

%future

The future devices discussed in this paper, gesture input and brain-computer interfaces are taking an even larger step away from previous computer input devices, in that they completely remove the computer as an entity in our workflow. Instead they let us focus on the task at hand without even reflecting about that there is actually a computer doing all the work in the background. This will allow the introduction of computer assisted processes in completely new parts of human lives, and has shown great prospects in the field of medicin. But this technology comes with important ethical issues; it could be used to access private information, possibly without the subject even being aware of the intrusion.

%special:

\todo{test-text}
The special-purpose input methods in current use make up a large selection of devices, 
\todo{/test-text}

\todo{REWRITE FROM HERE}
Computer interaction for people with different kinds of handicaps are often dependent on special purpose devices, depending on the nature of the disability. As computer output most commonly use displays, they only pose a problem for users with low eyesight or blindness, and the existing solutions are either tactile or hearing-based. The input devices on the other hand are much more diverse in their designs, as they need to adress a larger set of disabilities. Two main categories of special-purpose input methods are devices to aid people with reduced motor skills, and people with reduced or no eye-sight. Additionally, for persons with reduced motor skills, there are different methods available depending on how much motor skills is assumed.

%Hur bra är alternativen? Kan de ta del av viktig information i sin vardag? Hur hög medvetenhet finns det generellt? (tänker en webdesigner på sån nuförtiden?)
\todo{kommentar om alternativ}

For a person that has reduced motor skills in the hands only, the use of the standard keyboard and mouse can be difficult or impossible. There are solutions where a special designed keyboard and mouse designed for use with the feet can overcome this problem.
For persons that have more extensive reduction of motor skills the use of such devices might also be a problem. There are numerous different approaches to solve this. One of the most interesting is the use of eye movements to steer the computer. This approach makes use of cameras to track the users eye movements and act as a pointing device, which is also used to control an on-screen keyboard. 

For persons that have reduced eye sight the use of Braille input can be used to overcome computer use problems. There are numerous approaches with redesigned standard equipment like Braille keyboards, and speech feedback. There are also some completely different approaches that are more user centric, as an example the Perkinput system that uses the fingers on a touch surface (touch screen) to input Braille characters.

As illustrated by the case study of the Perkinput system, not all special purpose devices rely on specially designed hardware, but may rely on the software that handles these input devices. The Perkinput system allows the user to use the standard interface of the smart phone or tablet, but use a fundamentally different method of inputing text that is handled by the software that ease the process of data input into the system.

\section{Discussion}

% både tangentbordet och musen kom tidsbesparingskäl, istället för hålla på med hålremsor eller försöka styra grafisk information med tangenter.

Computers have gone from being blackboxes that take specially formated input and produce computer-formatted output that has to be interpreted by a human, to become more and more integrated with the regular life. It has also evolved from complex systems that require professional training to seemingly simple devices usable even by a child.

During this transformation different trends can be observed over time. The first trend was the process of making computer input faster while still retaining the underlying structures, the second was to make computers available to larger groups of people by making them simpler to use, and the third, and current trend, is the process of removing the computer as an entity and making it transparent to the user.

The first trend had no intentions to change the nature of computers at all, and did not do so in any extensive way. The introduction of keyboards and terminals were simply a way to speed up the process of computer input/output and were mostly technology-driven. 

The second trend, making computers accessible to larger groups of people, set in motion a large scale research and development cycle in both academic and commercial sectors which changed the focus from large main frame computers to desktop computers which introduced the user-centric approach to computers. The current trend in human-computer interface development displays an even larger focus on making the computers transparent to the users.

Since the progression towards a more human-centric development began, many different input devices have been introduced and are being used for tasks previously performed at a terminal-type computer. However, these systems do not necessarily replace the previous computer systems, instead the spectrum of computer interfaces in use is broadening. The concurrent use of different kinds of interfaces means that each individual interface is becoming more and more adapted to its context of use. The act of typing moderate or long pieces of text is tiring on a touchscreen device, being too slow and lacking the physical feedback of a real keyboard. On the other hand, inputing small amounts of data on the go is an activity that is made more cumbersome by a physical keyboard and it benefits heavily from the compact form of a touch screen device.

Examples of this concept can be seen across the field, where desktop and laptop computers are being replaced in areas were new technology is more suited, while they still remain in areas were new input systems have yet to outperform the previous systems. It is also evident that some of the current systems are outperforming eachother in different areas. An example is that voice recognition has become more and more used in simple query-based tasks, with technologies like Apple's Siri voice recognition system, where touchscreen-based systems in smartphones were previously prominent.

With the mainstream systems getting more and more diverse the division between special-purpose devices for persons with disabilities and standard devices will probably become less important, since every device will have a different design and ways of accessing the data. As this process continues, these devices will become just another device in the myriad of available devices.

In summary, computers have gone from a technology-centric design with a unified interface, to a large number of context-specific interfaces designed for more and more diverse tasks. With the these interfaces the computers themselves become such an integrated part of the process they are part of that they become invisible to the user. It is likely that this trend will continue and that computers with custom-designed interfaces will appear in almost every aspect of our lives. This will likely reduce the terminal-type computer, which has dominated the market for over 40 years, from an all-purpose system to the typewriter on which it is based. The gap created will be filled by better suited and transparent devices more adapted to perform the current and future tasks computers will be used for.