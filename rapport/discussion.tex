% !TEX root = rapport.tex

\section{Results}

%history:

The earliest computer interaction devices were not initially designed to be used with computers nor to make computers user-friendly. Instead previously existing technologies were directly taken and modified in order to function within their new setting. In contrast, the computer input devices in use at the time of this study are designed to make human interaction with computers as simple and easy as possible.

The typewriter-based \emph{terminal} was in a way the first input device specifically designed for computers. It removed many of the earlier problems associated with programming and running calculations, on the computers of that time. Although this device was very different from the earlier input devices, the underlying technology was still based on the previous input systems, punch cards. But still it abstracted away a lot of the technicalities that unnecessarily slowed the end user down. In contrast, the development of pointing devices was not related to the previous systems, but instead a result of the introduction of bitmap displays capable of displaying data in a different way than previous computer displays.

The big difference between these two important inventions is the type of impact they had. The terminal allowed the user to communicate with the computer more conveniently than before, but still in the same style. The mouse instead changed the way the user used the computer, being instrumental in the window-based systems of the late 80's and 90's. The impact of touchscreens is in a way similar to that of the mouse, in that it has lead to a new type of graphical user interfaces. Importantly, as these systems are mostly mobile devices, it has not only changed the way we use the computer but also how and when we can use them.

%Här är ett stycke som wrappar ihop de fyra tidigare. Den säger att fokus blivit mer och mer på användarvänligheten, och att de hör ihop. Man lärde sig att skiva text smidigt, men ville kunna lätt byta mellan olika "fönster" och flytta pekarens plats lättare. Touchpad, för att musen var för klumpig på laptop, touchscreen för att kunna ha bättre kontroll eller whatever. Jag vet inte riktigt varför man ville ha skiten, men vi skriver något.

%current:

The two current input methods discussed in this paper, speech recognition and touchscreens, are very different in their implementations, but the undelying idea is the same - to remove the hurdles in the interaction between man and machine. The touch screen has evolved as the next step from computer-mouse-keyboard interaction to direct interaction with the data the user is interested in. This makes the interaction much more direct and the learning curve for users is drastically reduced since the computer interacts with the user in the same way a physical object would do.

Speech recognition fulfills the same goal in easing the interraction between human and computer, but instead of doing it physically it is done by reacting to the spoken language in the same way the user would speak to another human. The ideas is to make the computer understand the user on the user's own terms. The development of speech recognition systems has focused on cognitive psychology and mathematic (statistical) models rather than on computer science, which reflects the previously mentioned change of focus towards user-friendliness.[reference?]

Despite recent advances in these fields and growth in popularity, they have not actually replaced the traditional terminal model of interaction, and likely they never will. These methods are used in other contexts than terminals, which may not be fitting in all circumstances. Particularily the act of typing moderate or long pieces of text is currently cumbersome using these methods; the touchscreen being to slow, and speech input too inaccurate.

%special:

Computer interaction for people with different kinds of handicaps utilize different methods, depending on the nature of the disability they are designed to address. As computer output most commonly use displays, they only pose a problem for users with low eye sight or blindness, and the existing solutions are either tactile or hearing-based. The input devices on the other hand are much more diverse in their designs, as they need to adress a larger set of disabilities. Two main categories of special-purpose input methods are devices to aid people with reduced motor skills, and people with reduced or no eye-sight.

Additionally, for persons with reduced motor skills, there are different methods available depending on how much motor skills is assumed.

%Är det som står nedan verkligen en diskussion? Vi borde tänka över vad en diskussion är i den här frågan.
%Hur bra är alternativen? Kan de ta del av viktig information i sin vardag? Hur hög medvetenhet finns det generellt? (tänker en webdesigner på sån nuförtiden?)

For a person that has reduced motor skills in the hands only, the use of the standard keyboard and mouse can be difficult or impossible. There are solutions where a special designed keyboard and mouse designed for use with the feet can overcome this problem.
For persons that have more extensivly reduced motor skills the use of such devices might also be a problem. There are numerous different approaches to solve this. One of the most interesting is the use of eye movements to steer the computer. This approach makes use of cameras to track the users eye movements and act as a pointing device, which is also used to control an on-screen keyboard. 

For persons that have reduced eye sight the use of braille input can be used to overcome computer use problems. There are numerous approaches with redesigned standard equipment like braille keyboards, and speech feedback. There are also some completely different approaches that are more user centric, as an example the Perkinput system that uses the fingers on a touch surface (touch screen) to input braille characters.

Prata om att inte alla lösningar är hårdvarubaserade, referera till case studyn.

\section{Discussion}

% både tangentbordet och musen kom tidsbesparingskäl, istället för hålla på med hålremsor eller försöka styra grafisk information med tangenter.