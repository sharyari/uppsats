\section{Special-Purpose Devices}

Many societies today use computers for the most diverse set of tasks. Computers have become such an integral part of our society that persons with handicaps that prevent them from using computers are at risk of getting left out of large parts of society and work opportunities. While the computers themselves are TODO of handicaps, the input and output devices present the user with accessibility problems. Persons with sight handicaps (TODO rätt ord?) are mostly concerned with output devices, since most output devices rely on eyesight, like bitmap displays or status LEDs. This can be circumvented with accessibility tools like Apple Inc's VoiceOver which reads the text and describes the interface shown on the display.

The input devices are much more diverse in their designs, because they need to adress a larger set of handicaps. The output devices mainly focus on sight, and somewhat on hearing (visual alerts -- hitta reference), while input devices have to focus on the whole spectrum. Two main categories of special-purpose input methods are devices to aid persons with reduced motor skills, and persons with reduced or no eye sight. 

Devices that aid persons with reduced motor skills [kommer i olika former] depending on what they are meant to concquer. In some cases the use of hand controlled devices is a problem, which has produced a lot of research on foot controlled devices. AUTHORS describe SOME PRODUCT that DOES SOMETHING WITH THE FEET. MORE INFORMATION. Another set of devices are designed to aid persons that have lost most motor skills. AUTHORS describe devices that are controlled using eyes or facial features that do not require any other movement. These devices are called gaze communication devices. Devices that use gaze communication, as described by AUTHORS, MULTIPLE\_REFERENCES are designed to allow persons who have extensive motor skill handicaps to use a specialised pointing device for controlling the computer. These devices rely on facial features and actions for controlling the cursor on the display, like nose tip and pupil movements for cursor movements and blinking for performing click actions.

Other special-purpose input devices are designed to help persons of other handicaps, and are designed for overcoming different accessibility problems. AUTHORS REFERENCE describe a special-purpose input method, designed mainly for smart phones and tablets which use touch screens, to input text using Input Finger Detection (IFD). While most smart phones today have some kind of input method to aid blind or low vision users, like Apple Inc's solution that reads the selected virtual key, and the user selects it by double tapping the display, the special purpose device allows the user to enter text in a much faster pace. AUTHORS input method is called Perkinput, and it allows the user to enter Braille letters directly by tapping fingers on the display. The Braille alphabet is built using a 2x3 bit matrix where each letter has certain bits "up" or "down". For two hand Braille input the user taps up to six fingers on the display, each finger representing a point in the Braille letters. The letters can be entered using one hand by first entering the left column with three fingers, followed by the right column with the same fingers. To enter the letter B, which can be seen as having the binary representation 110000, the user would press the two first fingers on the display to represent 110, and then do a one finger swipe to represent the remaining 000. The input rate of this special-purpose method is according to AUTHORS one handed Perkinput is evidentially faster than the iPhones standard input method, and two handed Perkinput is more than twice as fast. This comparison shows the difference a special-purpose device can make in the effectiveness of computer use related to adapted standard input methods. 

\begin{figure}[h!]

	\centering

\begin{tabular}{c c c}

$
\begin{array}{cc}
\bullet & \circ \\
\bullet & \circ \\
\circ & \circ \end{array}
$

&

$
\left[ \begin{array}{cc}
1 & 0 \\
1 & 0 \\
0 & 0 \end{array} \right]
$ 

&

110000 \\

(a) & (b) & (c)

\end{tabular}


	\caption{Three representations of the Braille letter B; the regular representation (a), a matrix representation (b), and a binary representation (c).}

\end{figure}

