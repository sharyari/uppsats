\section{Special-Purpose Devices}

Computers today are used for the most diverse set of tasks. They have become such an integral part of our society that people with handicaps that prevent them from using computers are at risk of getting left out of large parts of society and from work opportunities. While the computers themselves are indifferent towards handicaps, the input and output devices present the user with accessibility problems.

The obstacles for people with impaired vision are mostly concerned with output devices, since most output devices rely on graphics like bitmap displays or status LEDs. This can be circumvented with accessibility tools, such as the Apple Inc VoiceOver solution described below. But input methods may also present problems, for example the touchpads used in modern mobile phones.

For people with reduced motor skills, the obstacles lie mainly in the use of input devices and come in many different degrees. The standard computer mouse  can for too sensitive for some users, whereas both the mouse and keyboard may be unusable. Below, some of the most prominently used special-purpose devices are described.

%[SKRIV OM BRAILLE OUTPUT DEVICES]
%http://en.wikipedia.org/wiki/File:Refreshable_Braille_display.jpg

\subsection{Alternative Input Methods}
Devices aimed towards aiding people with reduced motor skills come in different types, depending on which issue they are designed to conquer. In some cases the use of hand controlled devices is a problem, which has led to research on alternative keyboard solutions. In an article from 1997 Uni-Ing et. al. \cite{583209} describe a foot controlled keyboard for persons that only have one foot. It consists of an arched set of keys centered around the heel of the feet used for typing. This kind of input device is aimed at amputees and persons with cerebral palsy. The more current research focuses more on the use of on-screen virtual keyboards (REF: 06141407, p203-wandmacher) \todo{ref}. 

%Bild som beskriver detta

Another set of devices currently in research are designed to aid persons that have lost most of their motor skills. An example of these devices are devices controlled using eyes or facial features which do not require any other movement. These devices are called \emph{gaze communication devices}. Devices that use gaze communication, as described by Parmar\cite{ieee6398171} and Arai\cite{conf/itng/AraiM11a} are designed to allow persons who have extensive motor skill handicaps to use a specialised pointing device for controlling the computer. These devices rely on facial features and actions for controlling the cursor on the display, like nose tip and pupil movements for cursor movements and blinking for performing click actions.

%Bild som beskriver detta

Most smart devices today have some kind of input method to aid blind users or users with impaired vision. The VoiceOver solution from Apple reads the selected virtual key out loud. The user then selects it by double tapping on the display\cite{voiceover}.

[Beskriv att det som följer är en case study typ]

A special purpose device that allows the user to enter text in a much faster pace when handling touch screen devices, in comparison to VoiceOver is the Perkinput method\cite{azenkot}. The method allows the user to enter \emph{Braille}\footnote{Writing system used by the blind.} letters directly by tapping fingers on the display.

The Braille alphabet is built using a 2x3 \emph{bit\footnote{The elevations used in Braille letters are known as bits, but differ from the familiar bits used in computer systems.} matrix} where each letter has certain \emph{bits} "up" or "down", see figure \ref{fig:brailleexample}. In the Perkinput system, using two hand Braille input the user taps up to six fingers on the display - each finger representing a bit in the Braille letters. The letters can also be entered using only one hand by first entering the left column with three fingers, followed by the right column with the same fingers.

For entering letters using two hands, the user uses three fingers on each hand. Each finger represents one of the bits in the Braille letter. To enter the letter B, having binary representation 110000 using two hands, the user would press the first two fingers on the first hand only, representing only the first two bits as 1:s.

To enter the letter B, using one hand, the user would press the two first fingers to represent 110, and then do a one finger swipe to represent the remaining 000.

The input rate of this special-purpose method is evidentially faster than the iPhones standard input method\cite{azenkot}, and two handed Perkinput is more than twice as fast. This comparison shows the difference a special-purpose device can make in the effectiveness of computer use related to adapted standard input methods. 

\begin{figure}[h!]
\centering

\begin{tabular}{c c c}

$
\begin{array}{cc}
\bullet & \circ \\
\bullet & \circ \\
\circ & \circ \end{array}
$

&

$
\left[ \begin{array}{cc}
1 & 0 \\
1 & 0 \\
0 & 0 \end{array} \right]
$ 

&

110000 \\

(a) & (b) & (c)

\end{tabular}


\caption{Three representations of the Braille letter B; the regular representation (a), a matrix representation (b), and a binary representation (c).}
\label{fig:brailleexample}


\end{figure}

\subsection{Discussion}
Computer interaction for people with different kinds of handicaps require olika metoder TODO. The computer output commonly uses screens and status LED:s, which only poses a problem for users with low eye sight or blindness. In some cases, the use of sound output poses a problem for hearing impaired users, however these can often be solved with the help of visual cues [HITTA REFERENCE]. 

The input devices are much more diverse in their designs, because they need to adress a larger set of handicaps. Two main categories of special-purpose input methods are devices to aid persons with reduced motor skills, and persons with reduced or no eye-sight.



Prata om att inte alla lösningar är hårdvarubaserade, referera till case studyn.

