\section{Special-Purpose Devices}

Computer systems have become a big part of our society that persons that have handicaps that prevent them from using computers are at risk of getting TODO SAK. To get around this there are a number of input devices available for persons having different kind of handicaps. How they are designed are very different depending on the type of handicap they are supposed to overcome. 

Some devices, like EXAMPLE + REFERENCE are designed to work with the standard interfaces of todays computers, while others, like EXAMPLE + REFERENCES have a dedicated system that supports them. 

Some of the devices that are designed to interact with the standard operating systems of today are modified versions of the standard devices, like keyboards with Braille letters on the keys REFERENCE, while other devices have a design that is specially designed to aid persons with a specific handicap. While these devices solve the same basic problem, they can be very different in how they look and work. As an example, gaze communication devices, as described by AUTHORS [REFERENCE] are, like the mouse, pointing devices that are used for moving the cursor on the display. While the mouse uses the desk top to determine movements, the gaze communication devices use the pupils of the eyes for cursor movement, and blinking for "clicking".