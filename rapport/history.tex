\section{History}
In 1946, the ENIAC computer was completed for the United States Army and it has been argued whether ENIAC is to be seen as the first digital computer\cite{McCartney1999}, or if this title actually belongs to the ABC computer (1942)\cite{court}. Regardless, it is clear that the history of the underlying computer interfaces used is longer than the history of the computers themselves.

Already in the beginning of the 18th century Basile Bouchon started using perforated paper to control the textile looms used for weaving[citation needed]. To control the cords of a warp, thick paper rolls were punched with patterns of holes, each column corresponding to a cord. The cords were then raised or lowered, depending on whether the paper was punched or not. In this manner Bouchons machine managed to automatise part of the weaving process, and allowed for more complex weaving patterns. Although punched cards were first invented in the 18th century, they were used as means of interaction both by the ABC and the ENIAC right at the beginning of modern computer history, and saw continued use until NÄR? [citation needed].



\subsection{Herman Hollerith and the Census Problem}
By the end of the 19th century the Bureau of the Census in the United States were having a problem. The Census is the agency in charge of keeping records of the population, and due to heavy immigration, the amount of data and complexity of the system was rapidly increasing. At that time, the Census was performing a population count every ten years - a process taking years to complete. In 1889, the director of the Census advertised a competition for the 1890 census tabulation system. The competition involved tabulating the St. Louis population district information from the previous 1880 census. The winner of the competition was Herman Hollerith, a former employee of the census bureau, with his electric tabulating machine.[citation needed, chapter 4]

[short description of the machine].

Hollerith's tabulating machine was a success, and the contract was renewed also for the 1900 census. In order to expand the custormer base for his tabulating machines, Hollerith founded the Tabulating Machine Company in 1896. His company was one of three companies that in 1911 merged to form the Computing Tabulating Recording Company (CTR) - later renamed IBM.[citation IMBHISTORY]

[short description of later date machines]

Punched cards were the most commonly used data medium until the 1950s, and were commonly used until the mid 1980s when they had become obsolete by the \emph{computer terminal}.[citation needed]

\subsection{The Teleprinter and the Terminal}
In 1901, Donald Murray developed a \emph{tele-typewriter} or \emph{teleprinter}. His idea was to simplify the work of telegraph operators, by using a typewriter keyboard. The operator would type on a typewriter, where every letter corresponded to a pattern of holes punched into a punch card. The information on the punch card was then transfered over the already existing telegraph lines and reproduced at the receiving end, again on punch cards.[citation needed]

This method of typing to punch cards was used by the first computers, and are known as \emph{computer terminals}. The term computer terminal is not exclusive to punch card-based typewriters. Though punch cards were the main method of data entry and data storage, alternative methods began to emerge in the early 1960's. 

In the 1960's, output devices had become more advanced and in the end of the 1960's, the first attempts were made to replace printing with a \emph{computer monitor}. This method of combining a typewriter-style \emph{keyboard} with a monitor is known as a \emph{computer terminal}. One of the earliest computer terminals was the Datapoint 3300 by the Computer Terminal Corporation[citation needed]. Using the Datapoint 3300, the user could control a \emph{cursor} by moving it up, down, left and right on the screen, but the terminal had no microprocessor.

Microprocessor computers allowed the combination of program and working memory and it allowed the computer programs to read data from the memory itself. To input data into the memory, punch cards could still be used. However, since all the data was stored electronically the data could as well be entered direcly into the memory. This removed the need for the punched paper middle step. The data that was previously transmitted via telegraph lines could be used to connect the typewriter directly to the computer memory. The initial disadvantage with this method was that memory was limited and expensive. Thus, the use of punch cards declined in favour of the computer terminal, in step with computer memory becoming more accessible.[citation needed]

\subsection{Pointing devices}
The first computer \emph{mouse} was developed in 1965 at the Stanford Research Laboratory and is contributed to Doug Engelbart. Engelbart also proposed applications for the mouse, such as using multiple tiled \emph{windows}, which became widely used in early graphical computer interfaces. Although invented in the 1960's, it was not commercially available until 1981 when released as part of the XEROX Star system. [citation Meyers].
XEROX soon got competition by the Apple Lisa (1983) and the popular Apple 128K (1984).

% Bild på första musen
% Bild på xerox-musen eller mac-musen

The introduction of this device allowed the users to move a pointer on the display that could be used to interact with the computer. A ball underneath the mouse gave relative x and y coordinates when rolling the ball on a surface. The mouse laid the ground for the MWI (Mouse Window Icon) graphical interfaces used today[reference needed]. The graphical user interface was invented by XEROX, and the Star system is wildely recognised as the first commercially available product with a graphical interface.

%As the displays became smaller and smaller and with the introduction of flat screens the portable computers were introduced. To keep these computers portable the mouse had to be built into the computer itself. This lead to a different kind of pointing devices.

%The portable computer pointing device problem was initially solved by turning the mouse upside down and using a "pointing ball" (pointing ballen kan ha kommit tidigare).

%The pointing balls were cumbersome/too large/nånting and a new input method for portable computers were needed. 

Jag vet ingenting om styrplattor. 

% BJÖRNTEXT END

%http://en.wikipedia.org/wiki/File:Hollerith_punched_card.jpg
%http://en.wikipedia.org/wiki/File:Basile_Bouchon_1725_loom.jpg
%http://www-03.ibm.com/ibm/history/documents/pdf/1885-1969.pdf - IBMHISTORY
