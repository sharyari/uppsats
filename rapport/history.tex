\section{History}
In 1946, the ENIAC computer was completed for the United States Army and it has been argued whether ENIAC is to be seen as the first digital computer\cite{McCartney1999}, or if this title actually belongs to the ABC computer (1942)\cite{court}. Regardless, it is clear that the history of the underlying computer interfaces used is longer than the history of the computers themselves.

Already in the beginning of the 18th century Basile Bouchon started using perforated paper to control the textile looms used for weaving[citation needed]. To control the cords of a warp thick paper rolls were punched with patterns of holes, each column corresponding to a cord. The cords were then raised or lowered, depending on whether the paper was punched or not. In this manner Bouchons machine managed to automatise part of the weaving process, and allowed for more complex weaving patterns. Although punched cards were first invented in the 18th century, they were used as means of interaction both by the ABC and the ENIAC right at the beginning of modern computer history, and saw continued use until NÄR? [citation needed].



\section{Herman Hollerith and the Census Problem}
By the end of the 19th century the Bureau of the Census in the United States were having a problem. The Census is the agency in charge of keeping records of the population, and due to heavy immigration, the amount and complexity of the system was rapidly increasing. At that time, the Census was performing a population count every ten years - a process taking years to complete. In 1889, the director of the Census advertised a competition for the 1890 census tabulation system. The competition involved tabulating the St. Louis population district information from the previous 1880 census. The winner of the competition was Herman Hollerith, a former employee of the census bureau, with his electric tabulating machine.[citation needed, chapter 4]

Hollerith's tabulating machine was a success, and the contract was renewed also for the 1900 census. In order to expand the custormer base for his tabulating machines, Hollerith founded the Tabulating Machine Company in 1896. His company was one of three companies that in 1911 merged to form the Computing Tabulating Recording Company (CTR) - later renamed IBM.[citation IMBHISTORY]

In th


%http://en.wikipedia.org/wiki/File:Hollerith_punched_card.jpg
%http://en.wikipedia.org/wiki/File:Basile_Bouchon_1725_loom.jpg
%http://www-03.ibm.com/ibm/history/documents/pdf/1885-1969.pdf - IBMHISTORY
