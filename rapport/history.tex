\section{History}
In 1946, the ENIAC computer was completed for the United States Army and it has been argued whether ENIAC is to be seen as the first digital computer\cite{McCartney1999}, or if this title actually belongs to the ABC computer (1942)\cite{court}. Regardless, it is clear that the history of the underlying computer interfaces used is longer than the history of the computers themselves.

Already in the beginning of the 18th century Basile Bouchon started using perforated paper to control the textile looms used for weaving[citation needed]. To control the cords of a warp, thick paper rolls were punched with patterns of holes, each column corresponding to a cord. The cords were then raised or lowered, depending on whether the paper was punched or not. In this manner Bouchons machine managed to automatise part of the weaving process, and allowed for more complex weaving patterns. Although punched cards were first invented in the 18th century, they were used as means of interaction both by the ABC and the ENIAC right at the beginning of modern computer history, and saw continued use until NÄR? [citation needed].



\subsection{Herman Hollerith and the Census Problem}
By the end of the 19th century the Bureau of the Census in the United States were having a problem. The Census is the agency in charge of keeping records of the population, and due to heavy immigration, the amount of data and complexity of the system was rapidly increasing. At that time, the Census was performing a population count every ten years - a process taking years to complete. In 1889, the director of the Census advertised a competition for the 1890 census tabulation system. The competition involved tabulating the St. Louis population district information from the previous 1880 census. The winner of the competition was Herman Hollerith, a former employee of the census bureau, with his electric tabulating machine.[citation needed, chapter 4]

[short description of the machine].

Hollerith's tabulating machine was a success, and the contract was renewed also for the 1900 census. In order to expand the custormer base for his tabulating machines, Hollerith founded the Tabulating Machine Company in 1896. His company was one of three companies that in 1911 merged to form the Computing Tabulating Recording Company (CTR) - later renamed IBM.[citation IMBHISTORY]

[short description of later date machines]

Punched cards were the most commonly used data medium until the 1950s, and were commonly used until the mid 1980s when they had become obsolete by the \emph{computer terminal}.[citation needed]

\subsection{The Teleprinter and the Terminal}
In 1901, Donald Murray developed a \emph{tele-typewriter} or \emph{teleprinter}. His idea was to simplify the work of telegraph operators, by using a typewriter keyboard. The operator would type on a typewriter, where every letter corresponded to a pattern of holes punched into a punch card. The information on the punch card was then transfered over the already existing telegraph lines and reproduced at the receiving end, again on punch cards.[citation needed]

This method of typing to punch cards was used by the first computers, and are known as \emph{computer terminals}. The term computer terminal is not exclusive to punch card-based typewriters. Though punch cards were the main method of data entry and data storage, alternative methods began to emerge in the early 1960's. 

% BJÖRNS SLUMPTEXT NEDAN, DET KAN VARA RENA OCH SKÄRA LÖGNER:

\subsection{Keyboards}

The popularisation of the microprocessor computers combined the program and working memory and it allowed the computer programs to read data from the memory itself. 

Något kort om assembler?

To input data into the memory, punch cards could still be used, however, since all the data was stored electronically the data could be entered direcly into the memory. This removed the need for the punched paper middle step. Instead the data that was previously transmitted via telegraph lines could be used to connect the typewriter directly to the computer memory. 

The typewriter evolved into the keyboards still in use today. Meanwhile the output devices became more and more advanced and evolved. Bitmap displays became more and more available. The bitmap display allowed the computer to draw graphics on the display. To control graphics via keyboard was hard (kanske? nånting var det iaf) and it created a need for a input device that worked well with the current input devices.

\subsection{Pointing devices}

The first mouse was invented in TODO at Xerox (vill jag minnas) and it consisted of a simple box with two wheels on the bottom that could give relative x and y coordinates to the computer, and a set of three (?) buttons to initiate actions.
The introduction of this device allowed the users to move a pointer on the display that could be used to interact with the computer. 

The two wheels under the mouse were later changed to a moving ball under the mouse with the x and y wheels rolling against the ball inte mouse interior. This gave better accuracy? (kanske?)

This in turn laid the ground for the MWI GUI:s we are used to today (referens till nån MWI - Menu Window Icon-artikel från 80-talet?).

As the displays became smaller and smaller and with the introduction of flat screens the portable computers were introduced. To keep these computers portable the mouse had to be built into the computer itself. This lead to a different kind of pointing devices.

The portable computer pointing device problem was initially solved by turning the mouse upside down and using a "pointing ball" (pointing ballen kan ha kommit tidigare).

The pointing balls were cumbersome/too large/nånting and a new input method for portable computers were needed. 

Jag vet ingenting om styrplattor. 

% BJÖRNTEXT END

%http://en.wikipedia.org/wiki/File:Hollerith_punched_card.jpg
%http://en.wikipedia.org/wiki/File:Basile_Bouchon_1725_loom.jpg
%http://www-03.ibm.com/ibm/history/documents/pdf/1885-1969.pdf - IBMHISTORY
